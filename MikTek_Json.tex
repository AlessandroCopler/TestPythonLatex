\documentclass{article}
\usepackage[left=2.8cm,top=0.5cm,right=2.8cm,bottom=0.5cm]{geometry}


\usepackage{luacode}
\renewcommand{\rmdefault}{phv}
\renewcommand{\sfdefault}{phv}

\usepackage{amsmath,amssymb}
\usepackage[italian]{babel}

\usepackage{setspace}
\usepackage[utf8]{inputenc}
\begin{document}

\begin{luacode}
function read(file)
    local handler = io.open(file, "rb")
    local content = handler:read("*all")
    handler:close()
    return content
end

JSON = (loadfile "JSON.lua")()
local table = JSON:decode(read("recipes.json"))
local table2 = JSON:decode(read("data.json"))

\end{luacode}


\begin{center}
\LARGE{TRIBUNALE ORDINARIO DI} 
\begin{luacode}
 local table = JSON:decode(read("data.json"))   	  
	  tex.print(table['ufficio']['distretto'])
\end{luacode}
\\
\vspace*{0.5cm}
\normalsize{PROCEDIMENTI SPECIALI SOMMARI}\\
\vspace*{0.5cm}
\LARGE{Nota di iscrizione a ruolo\\
o\\
Nota di accompagnamento}\\
\end{center}

%a=inizio del ciclo
%b=fine del ciclo
%c=step di incremento di a 

%\begin{luacode}
%a=1;b=3;c=1
%for var=a,b,c do 
%	local table = JSON:decode(read("recipes.json"))
%	tex.print(table['recipe']['ingredients'][var]['item'])
%end
%\end{luacode}

%\begin{luacode}
% local i = 1
% local table = JSON:decode(read("recipes.json"))
%    while (table['recipe']['ingredients'][i]) do
%	  tex.print(table['recipe']['ingredients'][i]['item'])
%	  i = i + 1
%    end
%\end{luacode}

Per ricorrente	
\hspace{1.0cm}		
$\text{\rlap{}}\square$
Per reclamante\vspace{0.5cm}\\
\linespread{1.5}{Si chiede l’iscrizione al \textbf{RUOLO GENERALE DEGLI AFFARI CIVILI - PROCEDIMENTI SPECIALI SOMMARI} della seguente causa introdotta con:}\vspace{0.5cm}\\
$\text{\rlap{$\checkmark$}}\square$
(0) Ricorso
\hspace{1.5cm}		
$\text{\rlap{}}\square$
(0) Citazione 
\hspace{1.5cm}	
$\text{\rlap{}}\square$
(7) Reclamo\vspace*{0.5cm}\\
\tiny{\textbf{P}romosso da:}\\
%\begin{luacode}
% local i = 1
% local table = JSON:decode(read("recipes.json"))
%     while (table['recipe']['ingredients'][i]) do
%	  if i==1 then     	  
%	  tex.print(table['recipe']['ingredients'][i]['item'])
%	  tex.print(",")
%	  i=i+1
%	  else 
%	  if(table['recipe']['ingredients'][i+1]) then
%	  		tex.print(table['recipe']['ingredients'][i]['item'])
%	  		tex.print(",")
%	  		i = i + 1
%	  	else
%	  		tex.print(table['recipe']['ingredients'][i]['item'])
%	  		tex.print(".")
%	  		i=i+1
%	  	end
%	  end
%    end
%\end{luacode} 
\\
\textbf{C}\tiny{on L'}\textbf{A}\tiny{vv.} 
\begin{luacode}
 local i = 1
 local table = JSON:decode(read("data.json"))
     while (table['attore'][1]['avvocato'][i]) do    	  
	  tex.print(table['attore'][1]['avvocato'][i]['nome'])
	  tex.print(",")
	  i=i+1
     end
\end{luacode}
\\
\textbf{C}\tiny{\textbf{ONTRO}}
%\begin{luacode}
% local i = 1
% local table = JSON:decode(read("recipes.json"))
%    while (table['recipe']['ingredients'][i]) do
%	  if i==1 then     	  
%	  tex.print(table['recipe']['ingredients'][i]['item'])
%	  tex.print(",")
%	  i=i+1
%	  else 
%	  if(table['recipe']['ingredients'][i+1]) then
%	  		tex.print(table['recipe']['ingredients'][i]['item'])
%	  		tex.print(",")
%	  		i = i + 1
%	  	else
%	  		tex.print(table['recipe']['ingredients'][i]['item'])
%	  		tex.print(".")
%	  		i=i+1
%	  	end
%	  end
%    end
%\end{luacode}
\\
\textbf{C}\tiny{on L'}\textbf{A}\tiny{vv.} \vspace*{1cm}
%\begin{luacode}
% local i = 1
% local table = JSON:decode(read("recipes.json"))
%     while (table['recipe']['ingredients'][i]) do
%	  if i==1 then     	  
%	  tex.print(table['recipe']['ingredients'][i]['item'])
%	  tex.print(",")
%	  i=i+1
%	  else 
%	  if(table['recipe']['ingredients'][i+1]) then
%	  		tex.print(table['recipe']['ingredients'][i]['item'])
%	  		tex.print(",")
%	  		i = i + 1
%	  	else
%	  		tex.print(table['recipe']['ingredients'][i]['item'])
%	  		tex.print(".")
%	  		i=i+1
%	  	end
%	  end
%    end
%\end{luacode} 
\\
\normalsize$\text{\rlap{}}\square$ Valore della controversia\footnote{Il Valore e determinato ai sensi dell'art.9 Legge 23.12.1999 n. 488 }
%\begin{luacode}
%local table = JSON:decode(read("recipes.json"))
%tex.print(table['recipe']['cal'])
%\end{luacode}
%\\
Importo del contributo unificato\footnotemark[1]\footnote{Allegare ricevuta di versamento}
%\begin{luacode}
%local table = JSON:decode(read("recipes.json"))
%tex.print(table['recipe']['cal'])
%\end{luacode}
\vspace{1cm} \\
$\text{\rlap{}}\square$ Esenzione dal contributo unificato.\newpage
Data di comparizione 
%\begin{luacode}
%local table = JSON:decode(read("recipes.json"))
%tex.print(table['recipe']['protein'])
%tex.print("  ")
%
%\end{luacode} 
 Data di notifica 
%\begin{luacode}
%local table = JSON:decode(read("recipes.json"))
%tex.print(table['recipe']['protein'])
%tex.print("  ")
%\end{luacode} 
Oggetto e Codice  procedimento sommario \hrulefill  .....$ | $.....$ | $.....$ | $ \footnote{indicare oggetto e codice relativo tra quelli elencati in tabella}\\
Oggetto e Codice domanda di merito \hrulefill .....$ | $.....$ | $.....$ | $\footnotemark[3] \\

$\text{\rlap{}}\square$ prova
\end{document}