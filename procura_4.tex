\documentclass{article}
\usepackage[left=3cm,top=3cm,right=3cm,bottom=3cm]{geometry}
\usepackage{luacode}
\renewcommand{\rmdefault}{phv}
\renewcommand{\sfdefault}{phv}
\usepackage{amsmath,amssymb}

\usepackage{setspace}
\usepackage[utf8]{inputenc}
\usepackage[italian]{babel}
\begin{document}
\begin{luacode}

function read(file)
    local handler = io.open(file, "rb")
    local content = handler:read("*all")
    handler:close()
    return content
end
JSON = (loadfile "JSON.lua")()
local table = JSON:decode(read("data2.json"))
\end{luacode}


\textbf{PROCURA 4}\\
- A) CLIENTE SOCIETA' - B) CONTROPARTE PERSONA FISICA\\
\vspace*{0.3cm}\\
\begin{luacode}
 local i = 1
 local table = JSON:decode(read("data2.json"))
 
--tipoParte dovra contenere la posizione dell'attore

 local tipoParte=0
 
 --cerco la posizione della voce ATTORE
 while (table['soggetti'][i]['tipoParte'] ~= "ATTORE") do
  i=i+1
 end
 
 --quando esce dal while ho trovato la posizione e la salvo in tipoParte
 tipoParte=i
 --resetto la variabile i che viene riusata
 i=1
 --ciclo per scorrere tutti i soggetti ATTORI
 
 while(table['soggetti'][tipoParte]['soggetto'][i]) do
  tex.print("Io sottoscritto")
  tex.print(table['soggetti'][tipoParte]['soggetto'][i]['titolarePiva']['nome'])
  tex.print(table['soggetti'][tipoParte]['soggetto'][i]['titolarePiva']['denominazione'])
  tex.print("con C.F.")
  tex.print(table['soggetti'][tipoParte]['soggetto'][i]['identificativo'])
  tex.print(",")
  tex.print("nella mia qualita di legale rappresentante di")
  tex.print(table['soggetti'][tipoParte]['soggetto'][i]['denominazione'])
  tex.print(", con sede legale in")
  tex.print(table['soggetti'][tipoParte]['soggetto'][i]['via'])
  tex.print(table['soggetti'][tipoParte]['soggetto'][i]['civico'])
  tex.print("e P.IVA")
  tex.print(table['soggetti'][tipoParte]['soggetto'][i]['identificativo'])
  tex.print(".")
  
  i=i+1
 end
 --se piu di uno 
 if i == 2 then
   tex.print("Conferisco mandato")
   else
   tex.print("Conferiamo mandato")
 end

\end{luacode}
%\begin{luacode}
% local table = JSON:decode(read("data2.json"))   	  
%	  tex.print(table['soggetti'][2]['soggetto']['titolarePiva']['denominazione'])
%%\end{luacode}
%, con C.F. 
%\begin{luacode}
% local table = JSON:decode(read("data2.json"))   	  
%	  tex.print(table['soggetti'][2]['soggetto']['identificativo'])
%\end{luacode}
, anche in via tra loro disgiunta, agli avv.ti Francesco Valeri e Riccardo Marini del Foro di Brescia a rappresentare e difendere la predetta societ\`a nella procedura monitoria contro
\begin{luacode}
 local i = 1
 local table = JSON:decode(read("data2.json"))
 --cerco la posizione della voce CONVENUTO
 while (table['soggetti'][i]['tipoParte'] ~= "CONVENUTO") do
  i=i+1
 end
 --quando esce dal while ho trovato la posizione e la salvo in tipoParte
 tipoParte=i
 --resetto la variabile i che viene riusata
 i=1
 --ciclo per scorrere tutti i soggetti CONVENUTO
 while(table['soggetti'][tipoParte]['soggetto'][i]) do
  tex.print(table['soggetti'][tipoParte]['soggetto'][i]['nome'])
    tex.print(table['soggetti'][tipoParte]['soggetto'][i]['denominazione'])
    tex.print(", con C.F.")
    tex.print(table['soggetti'][tipoParte]['soggetto'][i]['identificativo'])
    tex.print(",")
  i=i+1
 end
    
\end{luacode}

in ogni stato e grado della procedura medesima, ivi inclusa la fase di eventuale opposizione e la fase di esecuzione forzata, conferendo agli stessi ogni pi\`u ampio potere inerente la procura ivi incluso quello di conciliare e transigere la lite, incassare somme e rilasciare quietanze, disconoscere scritture private, presentare istanze di fallimento, farsi sostituire da altri professionisti, nominare consulenti di parte, dando sin d'ora per rato e valido il loro operato ed eleggendo domicilio speciale 
\begin{luacode}
 local i = 1
 local table = JSON:decode(read("data2.json"))
 
--tipoParte dovra contenere la posizione dell'attore

 local tipoParte=0
 
 --cerco la posizione della voce ATTORE
 while (table['soggetti'][i]['tipoParte'] ~= "ATTORE") do
  i=i+1
 end
 
 --quando esce dal while ho trovato la posizione e la salvo in tipoParte
 tipoParte=i
 --resetto la variabile i che viene riusata
 i=1

 
  while(table['soggetti'][tipoParte]['soggetto'][i]) do
  i=i+1
  end
  
  if i == 2 then
  tex.print("della	")
  else
  tex.print("delle	")
  end
\end{luacode}
societ\'a
\begin{luacode}
 local i = 1
 local table = JSON:decode(read("data2.json"))
 
--tipoParte dovra contenere la posizione dell'attore

 local tipoParte=0
 
 --cerco la posizione della voce ATTORE
 while (table['soggetti'][i]['tipoParte'] ~= "ATTORE") do
  i=i+1
 end
 
 --quando esce dal while ho trovato la posizione e la salvo in tipoParte
 tipoParte=i
 --resetto la variabile i che viene riusata
 i=1
 --ciclo per scorrere tutti i soggetti ATTORI
 while(table['soggetti'][tipoParte]['soggetto'][i]) do
 tex.print(table['soggetti'][tipoParte]['soggetto'][i]['denominazione'])
 tex.print(",	")
 i=i+1
 end
\end{luacode}
presso il loro studio in Brescia, c.so G. Matteotti, 54.\\
\vspace*{0.3cm}\\
In fede\\
\vspace*{0.3cm}\\
\hspace*{12cm}E' autentica 
\end{document}